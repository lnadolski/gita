% !TeX program = XeLaTeX
% !TeX root = gitabook.tex
\sect{एकादशोऽध्यायः}
{अर्जुन उवाच}\nopagebreak[4]
\twolineshloka
{मदनुग्रहाय परमं गुह्यमध्यात्मसंज्ञितम्}
{यत्त्वयोक्तं वचस्तेन मोहोऽयं विगतो मम}% .. 11-1

\twolineshloka
{भवाप्ययौ हि भूतानां श्रुतौ विस्तरशो मया}
{त्वत्तः कमलपत्राक्ष माहात्म्यमपि चाव्ययम्}% .. 11-2

\twolineshloka
{एवमेतद्यथाऽऽत्थ त्वमात्मानं परमेश्वर}
{द्रष्टुमिच्छामि ते रूपमैश्वरं पुरुषोत्तम}% .. 11-3

\twolineshloka
{मन्यसे यदि तच्छक्यं मया द्रष्टुमिति प्रभो}
{योगेश्वर ततो मे त्वं दर्शयऽऽत्मानमव्ययम्}% .. 11-4

{श्रीभगवानुवाच}
\twolineshloka
{पश्य मे पार्थ रूपाणि शतशोऽथ सहस्रशः}
{नानाविधानि दिव्यानि नानावर्णाकृतीनि च}% .. 11-5

\twolineshloka
{पश्यऽऽदित्यान् वसून् रुद्रानश्विनौ मरुतस्तथा}
{बहून्यदृष्टपूर्वाणि पश्यऽऽश्चर्याणि भारत}% .. 11-6

\twolineshloka
{इहैकस्थं जगत्कृत्स्नं पश्यऽऽद्य सचराचरम्}
{मम देहे गुडाकेश यच्चान्यद्-द्रष्टुमिच्छसि}% .. 11-7

\twolineshloka
{न तु मां शक्यसे द्रष्टुमनेनैव स्वचक्षुषा}
{दिव्यं ददामि ते चक्षुः पश्य मे योगमैश्वरम्}% .. 11-8

{सञ्जय उवाच}\nopagebreak[4]
\twolineshloka
{एवमुक्त्वा ततो राजन् महायोगेश्वरो हरिः}
{दर्शयामास पार्थाय परमं रूपमैश्वरम्}% .. 11-9

\twolineshloka
{अनेकवक्त्रनयनमनेकाद्भुतदर्शनम्}
{अनेकदिव्याभरणं दिव्यानेकोद्यतायुधम्}% .. 11-10

\twolineshloka
{दिव्यमाल्याम्बरधरं दिव्यगन्धानुलेपनम्}
{सर्वाश्चर्यमयं देवमनन्तं विश्वतोमुखम्}% .. 11-11

\twolineshloka
{दिवि सूर्यसहस्रस्य भवेद्युगपदुत्थिता}
{यदि भाः सदृशी सा स्याद्भासस्तस्य महात्मनः}% .. 11-12

\twolineshloka
{तत्रैकस्थं जगत्कृत्स्नं प्रविभक्तमनेकधा}
{अपश्यद्देवदेवस्य शरीरे पाण्डवस्तदा}% .. 11-13

\twolineshloka
{ततः स विस्मयाविष्टो हृष्टरोमा धनञ्जयः}
{प्रणम्य शिरसा देवं कृताञ्जलिरभाषत}% .. 11-14

{अर्जुन उवाच}\nopagebreak[4]
\fourlineindentedshloka
{पश्यामि देवांस्तव देव देहे}
{सर्वांस्तथा भूतविशेषसङ्घान्}
{ब्रह्माणमीशं कमलासनस्थम्}
{ऋषींश्च सर्वानुरगांश्च दिव्यान्}% .. 11-15

\fourlineindentedshloka
{अनेकबाहूदरवक्त्रनेत्रम्}
{पश्यामि त्वां सर्वतोऽनन्तरूपम्}
{नान्तं न मध्यं न पुनस्तवऽऽदिम्}
{पश्यामि विश्वेश्वर विश्वरूप}% .. 11-16

\fourlineindentedshloka
{किरीटिनं गदिनं चक्रिणं च}
{तेजोराशिं सर्वतो दीप्तिमन्तम्}
{पश्यामि त्वां दुर्निरीक्ष्यं समन्ताद्-}
{दीप्तानलार्कद्युतिमप्रमेयम्}% .. 11-17

\fourlineindentedshloka
{त्वमक्षरं परमं वेदितव्यम्}
{त्वमस्य विश्वस्य परं निधानम्}
{त्वमव्ययः शाश्वतधर्मगोप्ता}
{सनातनस्त्वं पुरुषो मतो मे}% .. 11-18

\fourlineindentedshloka
{अनादिमध्यान्तमनन्तवीर्यम्}
{अनन्तबाहुं शशिसूर्यनेत्रम्}
{पश्यामि त्वां दीप्तहुताशवक्त्रम्}
{स्वतेजसा विश्वमिदं तपन्तम्}% .. 11-19

\fourlineindentedshloka
{द्यावापृथिव्योरिदमन्तरं हि}
{व्याप्तं त्वयैकेन दिशश्च सर्वाः}
{दृष्ट्वाऽद्भुतं रूपमुग्रं तवेदम्}
{लोकत्रयं प्रव्यथितं महात्मन्}% .. 11-20

\fourlineindentedshloka
{अमी हि त्वां सुरसङ्घा विशन्ति}
{केचिद्भीताः प्राञ्जलयो गृणन्ति}
{स्वस्तीत्युक्त्वा महर्षिसिद्धसङ्घाः}
{स्तुवन्ति त्वां स्तुतिभिः पुष्कलाभिः}% .. 11-21

\fourlineindentedshloka
{रुद्रादित्या वसवो ये च साध्याः}
{विश्वेश्विनौ मरुतश्चोष्मपाश्च}
{गन्धर्वयक्षासुरसिद्धसङ्घाः}
{वीक्षन्ते त्वां विस्मिताश्चैव सर्वे}% .. 11-22

\fourlineindentedshloka
{रूपं महत्ते बहुवक्त्रनेत्रम्}
{महाबाहो बहुबाहूरुपादम्}
{बहूदरं बहुदंष्ट्राकरालम्}
{दृष्ट्वा लोकाः प्रव्यथितास्तथाऽहम्}% .. 11-23

\fourlineindentedshloka
{नभःस्पृशं दीप्तमनेकवर्णम्}
{व्यात्ताननं दीप्तविशालनेत्रम्}
{दृष्ट्वा हि त्वां प्रव्यथितान्तरात्मा}
{धृतिं न विन्दामि शमं च विष्णो}% .. 11-24

\fourlineindentedshloka
{दंष्ट्राकरालानि च ते मुखानि}
{दृष्ट्वैव कालानलसन्निभानि}
{दिशो न जाने न लभे च शर्म}
{प्रसीद देवेश जगन्निवास}% .. 11-25

\fourlineindentedshloka
{अमी च त्वां धृतराष्ट्रस्य पुत्राः}
{सर्वे सहैवावनिपालसङ्घैः}
{भीष्मो द्रोणः सूतपुत्रस्तथाऽसौ}
{सहास्मदीयैरपि योधमुख्यैः}% .. 11-26

\fourlineindentedshloka
{वक्त्राणि ते त्वरमाणा विशन्ति}
{दंष्ट्राकरालानि भयानकानि}
{केचिद्विलग्ना दशनान्तरेषु}
{सन्दृश्यन्ते चूर्णितैरुत्तमाङ्गैः}% .. 11-27

\fourlineindentedshloka
{यथा नदीनां बहवोऽम्बुवेगाः}
{समुद्रमेवाभिमुखा द्रवन्ति}
{तथा तवामी नरलोकवीरा}
{विशन्ति वक्त्राण्यभिविज्वलन्ति}% .. 11-28

\fourlineindentedshloka
{यथा प्रदीप्तं ज्वलनं पतङ्गाः}
{विशन्ति नाशाय समृद्धवेगाः}
{तथैव नाशाय विशन्ति लोकाः}
{तवापि वक्त्राणि समृद्धवेगाः}% .. 11-29

\fourlineindentedshloka
{लेलिह्यसे ग्रसमानः समन्तात्}
{लोकान् समग्रान् वदनैर्ज्वलद्भिः}
{तेजोभिरापूर्य जगत् समग्रम्}
{भासस्तवोग्राः प्रतपन्ति विष्णो}% .. 11-30

\fourlineindentedshloka
{आख्याहि मे को भवानुग्ररूपो-}
{नमोऽस्तु ते देववर प्रसीद}
{विज्ञातुमिच्छामि भवन्तमाद्यम्}
{न हि प्रजानामि तव प्रवृत्तिम्}% .. 11-31

%\clearpage
{श्रीभगवानुवाच}
\fourlineindentedshloka
{कालोऽस्मि लोकक्षयकृत् प्रवृद्धो-}
{लोकान् समाहर्तुमिह प्रवृत्तः}
{ऋतेऽपि त्वां न भविष्यन्ति सर्वे}
{येऽवस्थिताः प्रत्यनीकेषु योधाः}% .. 11-32

\fourlineindentedshloka
{तस्मात् त्वमुत्तिष्ठ यशो लभस्व}
{जित्वा शत्रून् भुङ्क्ष्व राज्यं समृद्धम्}
{मयैवैते निहताः पूर्वमेव}
{निमित्तमात्रं भव सव्यसाचिन्}% .. 11-33

\fourlineindentedshloka
{द्रोणं च भीष्मं च जयद्रथं च}
{कर्णं तथाऽन्यानपि योधवीरान्}
{मया हतांस्त्वं जहि मा व्यथिष्ठा}
{युध्यस्व जेतासि रणे सपत्नान्}% .. 11-34

{सञ्जय उवाच}
\fourlineindentedshloka
{एतच्छ्रुत्वा वचनं केशवस्य}
{कृताञ्जलिर्वेपमानः किरीटी}
{नमस्कृत्वा भूय एवऽऽह कृष्णम्}
{सगद्गदं भीतभीतः प्रणम्य}% .. 11-35

%\clearpage
{अर्जुन उवाच}
\fourlineindentedshloka
{स्थाने हृषीकेश तव प्रकीर्त्या}
{जगत् प्रहृष्यत्यनुरज्यते च}
{रक्षांसि भीतानि दिशो द्रवन्ति}
{सर्वे नमस्यन्ति च सिद्धसङ्घाः}% .. 11-36

\fourlineindentedshloka
{कस्माच्च ते न नमेरन् महात्मन्}
{गरीयसे ब्रह्मणोऽप्यादिकर्त्रे}
{अनन्त देवेश जगन्निवास}
{त्वमक्षरं सदसत्तत्परं यत्}% .. 11-37

\fourlineindentedshloka
{त्वमादिदेवः पुरुषः पुराणः}
{त्वमस्य विश्वस्य परं निधानम्}
{वेत्ताऽसि वेद्यं च परं च धाम}
{त्वया ततं विश्वमनन्तरूप}% .. 11-38

\fourlineindentedshloka
{वायुर्यमोऽग्निर्वरुणः शशाङ्कः}
{प्रजापतिस्त्वं प्रपितामहश्च}
{नमो नमस्तेऽस्तु सहस्रकृत्वः}
{पुनश्च भूयोऽपि नमो नमस्ते}% .. 11-39

\fourlineindentedshloka
{नमः पुरस्तादथ पृष्ठतस्ते}
{नमोऽस्तु ते सर्वत एव सर्व}
{अनन्तवीर्यामितविक्रमस्त्वम्}
{सर्वं समाप्नोषि ततोऽसि सर्वः}% .. 11-40

\fourlineindentedshloka
{सखेति मत्वा प्रसभं यदुक्तम्}
{हे कृष्ण हे यादव हे सखेति}
{अजानता महिमानं तवेदम्}
{मया प्रमादात्प्रणयेन वाऽपि}% .. 11-41

\fourlineindentedshloka
{यच्चावहासार्थमसत्कृतोऽसि}
{विहारशय्यासनभोजनेषु}
{एकोऽथवाऽप्यच्युत तत् समक्षम्}
{तत् क्षामये त्वामहमप्रमेयम्}% .. 11-42

\fourlineindentedshloka
{पिताऽसि लोकस्य चराचरस्य}
{त्वमस्य पूज्यश्च गुरुर्गरीयान्}
{न त्वत्समोऽस्त्यभ्यधिकः कुतोऽन्यो}
{लोकत्रयेऽप्यप्रतिमप्रभाव}% .. 11-43

\fourlineindentedshloka
{तस्मात् प्रणम्य प्रणिधाय कायम्}
{प्रसादये त्वामहमीशमीड्यम्}
{पितेव पुत्रस्य सखेव सख्युः}
{प्रियः प्रियायार्हसि देव सोढुम्}% .. 11-44

\fourlineindentedshloka
{अदृष्टपूर्वं हृषितोऽस्मि दृष्ट्वा}
{भयेन च प्रव्यथितं मनो मे}
{तदेव मे दर्शय देव रूपम्}
{प्रसीद देवेश जगन्निवास}% .. 11-45

\fourlineindentedshloka
{किरीटिनं गदिनं चक्रहस्तम्}
{इच्छामि त्वां द्रष्टुमहं तथैव}
{तेनैव रूपेण चतुर्भुजेन}
{सहस्रबाहो भव विश्वमूर्ते}% .. 11-46

{श्रीभगवानुवाच}
\fourlineindentedshloka
{मया प्रसन्नेन तवार्जुनेदम्}
{रूपं परं दर्शितमात्मयोगात्}
{तेजोमयं विश्वमनन्तमाद्यम्}
{यन्मे त्वदन्येन न दृष्टपूर्वम्}% .. 11-47

\fourlineindentedshloka
{न वेदयज्ञाध्ययनैर्न दानैः}
{न च क्रियाभिर्न तपोभिरुग्रैः}
{एवं रूपः शक्य अहं नृलोके}
{द्रष्टुं त्वदन्येन कुरुप्रवीर}% .. 11-48

\fourlineindentedshloka
{मा ते व्यथा मा च विमूढभावो}
{दृष्ट्वा रूपं घोरमीदृङ्ममेदम्}
{व्यपेतभीः प्रीतमनाः पुनस्त्वम्}
{तदेव मे रूपमिदं प्रपश्य}% .. 11-49

{सञ्जय उवाच}
\fourlineindentedshloka
{इत्यर्जुनं वासुदेवस्तथोक्त्वा}
{स्वकं रूपं दर्शयामास भूयः}
{आश्वासयामास च भीतमेनम्}
{भूत्वा पुनः सौम्यवपुर्महात्मा}% .. 11-50

{अर्जुन उवाच}
\twolineshloka
{दृष्ट्वेदं मानुषं रूपं तव सौम्यं जनार्दन}
{इदानीमस्मि संवृत्तः सचेताः प्रकृतिं गतः}% .. 11-51

{श्रीभगवानुवाच}
\twolineshloka
{सुदुर्दर्शमिदं रूपं दृष्टवानसि यन्मम}
{देवा अप्यस्य रूपस्य नित्यं दर्शनकाङ्क्षिणः}% .. 11-52

\twolineshloka
{नाहं वेदैर्न तपसा न दानेन न चेज्यया}
{शक्य एवंविधो द्रष्टुं दृष्टवानसि मां यथा}% .. 11-53

\twolineshloka
{भक्त्या त्वनन्यया शक्यम् अहमेवंविधोऽर्जुन}
{ज्ञातुं द्रष्टुं च तत्त्वेन प्रवेष्टुं च परन्तप}% .. 11-54

\twolineshloka
{मत्कर्मकृन्मत्परमो मद्भक्तः सङ्गवर्जितः}
{निर्वैरः सर्वभूतेषु यः स मामेति पाण्डव}% .. 11-55
{॥ॐ तत्सदिति श्रीमद्भगवद्गीतासूपनिषत्सु ब्रह्मविद्यायां योगशास्त्रे श्रीकृष्णार्जुनसंवादे विश्वरूपदर्शनयोगो नाम एकादशोऽध्यायः॥}
% !TeX program = XeLaTeX
% !TeX root = gitabook.tex
\sect{पञ्चदशोऽध्यायः}
{श्रीभगवानुवाच}
\twolineshloka
{ऊर्ध्वमूलमधःशाखमश्वत्थं प्राहुरव्ययम्}
{छन्दांसि यस्य पर्णानि यस्तं वेद स वेदवित्}% .. 15-1

\fourlineindentedshloka
{अधश्चोर्ध्वं प्रसृतास्तस्य शाखाः}
{गुणप्रवृद्धा विषयप्रवालाः}
{अधश्च मूलान्यनुसन्ततानि}
{कर्मानुबन्धीनि मनुष्यलोके}% .. 15-2

\fourlineindentedshloka
{न रूपमस्येह तथोपलभ्यते}
{नान्तो न चऽऽदिर्न च सम्प्रतिष्ठा}
{अश्वत्थमेनं सुविरूढमूलम्}
{असङ्गशस्त्रेण दृढेन छित्त्वा}% .. 15-3

\fourlineindentedshloka
{ततः पदं तत् परिमार्गितव्यम्}
{यस्मिन् गता न निवर्तन्ति भूयः}
{तमेव चऽऽद्यं पुरुषं प्रपद्ये}
{यतः प्रवृत्तिः प्रसृता पुराणी}% .. 15-4

\fourlineindentedshloka
{निर्मानमोहा जितसङ्गदोषाः}
{अध्यात्मनित्या विनिवृत्तकामाः}
{द्वन्द्वैर्विमुक्ताः सुखदुःखसंज्ञैः}
{गच्छन्त्यमूढाः पदमव्ययं तत्}% .. 15-5

\twolineshloka
{न तद्भासयते सूर्यो न शशाङ्को न पावकः}
{यद्गत्वा न निवर्तन्ते तद्धाम परमं मम}% .. 15-6

\twolineshloka
{ममैवांशो जीवलोके जीवभूतः सनातनः}
{मनःषष्ठानीन्द्रियाणि प्रकृतिस्थानि कर्षति}% .. 15-7

\twolineshloka
{शरीरं यदवाप्नोति यच्चाप्युत्क्रामतीश्वरः}
{गृहीत्वैतानि संयाति वायुर्गन्धानिवाशयात्}% .. 15-8

\twolineshloka
{श्रोत्रं चक्षुः स्पर्शनं च रसनं घ्राणमेव च}
{अधिष्ठाय मनश्चायं विषयानुपसेवते}% .. 15-9

\twolineshloka
{उत्क्रामन्तं स्थितं वाऽपि भुञ्जानं वा गुणान्वितम्}
{विमूढा नानुपश्यन्ति पश्यन्ति ज्ञानचक्षुषः}% .. 15-10

\twolineshloka
{यतन्तो योगिनश्चैनं पश्यन्त्यात्मन्यवस्थितम्}
{यतन्तोऽप्यकृतात्मानो नैनं पश्यन्त्यचेतसः}% .. 15-11

\twolineshloka
{यदाऽऽदित्यगतं तेजो जगद्भासयतेऽखिलम्}
{यच्चन्द्रमसि यच्चाग्नौ तत्तेजो विद्धि मामकम्}% .. 15-12

\twolineshloka
{गामाविश्य च भूतानि धारयाम्यहमोजसा}
{पुष्णामि चौषधीः सर्वाः सोमो भूत्वा रसात्मकः}% .. 15-13

\twolineshloka
{अहं वैश्वानरो भूत्वा प्राणिनां देहमाश्रितः}
{प्राणापानसमायुक्तः पचाम्यन्नं चतुर्विधम्}% .. 15-14

\fourlineindentedshloka
{सर्वस्य चाहं हृदि सन्निविष्टो-}
{मत्तः स्मृतिर्ज्ञानमपोहनं च}
{वेदैश्च सर्वैरहमेव वेद्यो-}
{वेदान्तकृद्वेदविदेव चाहम्}% .. 15-15

\twolineshloka
{द्वाविमौ पुरुषौ लोके क्षरश्चाक्षर एव च}
{क्षरः सर्वाणि भूतानि कूटस्थोऽक्षर उच्यते}% .. 15-16

\twolineshloka
{उत्तमः पुरुषस्त्वन्यः परमात्मेत्युदाहृतः}
{यो लोकत्रयमाविश्य बिभर्त्यव्यय ईश्वरः}% .. 15-17

\twolineshloka
{यस्मात् क्षरमतीतोऽहम् अक्षरादपि चोत्तमः}
{अतोऽस्मि लोके वेदे च प्रथितः पुरुषोत्तमः}% .. 15-18

\twolineshloka
{यो मामेवमसम्मूढो जानाति पुरुषोत्तमम्}
{स सर्वविद्भजति मां सर्वभावेन भारत}% .. 15-19

\twolineshloka
{इति गुह्यतमं शास्त्रमिदमुक्तं मयाऽनघ}
{एतद्-बुद्‌ध्वा बुद्धिमान् स्यात् कृतकृत्यश्च भारत}% .. 15-20
{॥ॐ तत्सदिति श्रीमद्भगवद्गीतासूपनिषत्सु ब्रह्मविद्यायां योगशास्त्रे श्रीकृष्णार्जुनसंवादे पुरुषोत्तमयोगो नाम पञ्चदशोऽध्यायः॥}
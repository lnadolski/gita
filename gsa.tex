% !TeX program = XeLaTeX
% !TeX root = gitabook.tex
\sect{गीतार्थसङ्ग्रहः}

\twolineshloka*
{यत्पदाम्भोरुहध्यानविध्वस्ताशेषकल्मषः}
{वस्तुतामुपयातोऽहं यामुनेयं नमामि तम्}

\twolineshloka
{स्वधर्मज्ञानवैरग्यसाध्यभक्त्येकगोचरः}
{नारायणः परं ब्रह्म गीताशास्त्रे समीरितः}

\twolineshloka
{ज्ञानकर्मात्मिके नेष्टे योगलक्ष्ये सुसंस्कृते}
{आत्मानुभोतिसिद्‌ध्यर्थे पूर्वषट्केन चोदिते}

\twolineshloka
{मध्यमे भगवत्तत्त्वयाऽथऽऽत्म्यावाप्तिसिद्धये}
{ज्ञानकर्माभिनिर्वर्त्यो भक्तियोगः प्रकीर्तितः}

\twolineshloka
{प्रधानपुरुषव्यक्तसर्वेश्वरविवेचनम्}
{कर्मधीर्भक्तिरित्यादि पोर्वसेषोऽन्तिमोदितः}

\twolineshloka
{अस्थानस्नेहकार्पण्यधर्माधर्मधियाऽऽकुलम्}
{पार्थं प्रपन्नमुद्दिश्य शास्त्रावतरणं कृतम्}

\twolineshloka
{नित्यात्मासङ्गकर्मेहागोचरा सङ्ख्ययोगधीः}
{द्वितीये स्थितधीलक्षा प्रोक्ता तन्मोहशान्तये}

\twolineshloka
{असक्त्या लोकरक्षायै गुणेष्वारोप्य कर्तृताम्}
{सर्वेश्वरे वा न्यस्योक्ता तृतीये कर्मकार्यता}

\twolineshloka
{प्रसङ्गात् स्वस्वभावोक्तिः कर्मणोऽकर्मताऽस्य च}
{भेदा ज्ञानस्य माहत्म्यं चतुर्थाध्याय उच्यते}

\twolineshloka
{कर्मयोगस्य सौकर्यं शैघ्र्यं काश्चन तद्विधाः}
{ब्रह्मज्ञानप्रकारश्च पञ्चमाध्याय उच्यते}

\twolineshloka
{योगाभ्यासविधिर्योगी चतुर्धा योगसाधनम्}
{योगसिद्धिः स्वयोगस्य पारम्यं षष्ट उच्यते}

\twolineshloka
{स्वयाऽथऽऽत्म्यं प्रकृत्यास्य तिरोधिः शरणागतिः}
{भक्तभेदः प्रबुद्धस्य श्रैष्ट्यं सप्तम उच्यते}

\twolineshloka
{ऐश्वर्याऽक्षरयाऽथऽऽत्म्यभगवच्चरणार्थिनाम्}
{वेध्योपादेयभावानामष्टमे भेद उच्यते}

\twolineshloka
{स्वमाहात्म्यं मनुष्यत्वे परत्वं च महात्मनाम्}
{विशेषो नवमे योगो भक्तिरूपः प्रकीर्तितः}

\twolineshloka
{स्वकल्याणगुणानन्त्यकृत्स्नस्वाधीनतामतिः}
{भक्त्युत्पत्तिविवृद्‌ध्यर्था विस्तीर्णा दशमोदिता}

\twolineshloka
{एकादशे स्वयाऽथऽऽत्म्यसाक्षात्कारावलोकनम्}
{दत्तमुक्तं विदिप्राप्त्योर्भक्त्येकोपायता तथा}

\twolineshloka
{भक्तेः श्राष्ट्यमुपायोक्तिरशक्तस्यऽऽत्मनिष्टता}
{तत्प्रकारस्त्वतिप्रीतिः भक्तेर्द्वादश उच्यते}

\twolineshloka
{देहस्वरूपमात्माप्तिहेतुरात्मविशोधनम्}
{बन्धहेतुर्विवेकश्च त्रयोदश उदीर्यते}

\twolineshloka
{गुणबन्धविधा तेषां कर्तृत्वं तन्निवर्तनम्}
{गतित्रयस्वमूलत्वं चतुर्दश उदीर्यते}

\twolineshloka
{अचिन्मिश्राद्विशुद्धाच्च चेतनात् पुरुषोत्तमः}
{व्यापनाद्भरनात् स्वम्यदन्यः पञ्चदशोदितः}

\twolineshloka
{देवासुरविभगोक्तिपूर्विका शास्त्रवश्यता}
{तत्त्वनुष्टानविज्ञानस्थेम्ने षोडश उच्यते}

\twolineshloka
{अशास्त्रमासुरं कृत्स्नं शास्त्रीयं गुणतः पृथक्}
{लक्षणं शास्त्रसिद्धस्य त्रिधा सप्तदशोदितम्}

\twolineshloka
{ईश्वरे कर्तृताबुद्धिः सत्त्वोपादेयताऽन्तिमे}
{स्वकर्मपरिणामश्च शास्त्रसारार्थ उच्यते}

\twolineshloka
{कर्मयोगस्तपस्तीर्थदानयज्ञादिसेवनम्}
{ज्ञानयोगो जितस्वान्तैः परिशुद्धात्मनि स्थितिः}

\twolineshloka
{भक्तियोगः परैकान्तप्रीत्या ध्यानादिषु स्थितिः}
{त्रयाणामपि योगानां त्रिभिरन्योन्यसङ्गमः}

\twolineshloka
{नित्यनैमित्तिकानां च पराराधनरूपिणाम्}
{आत्मधष्टेस्रयोऽप्येते योगद्वारेण साधकाः}

\twolineshloka
{निरस्तनिखिलाज्ञानो दृष्ट्वाऽऽत्मानं परानुगम्}
{प्रतिलभ्य परां भक्तिं तयैवऽऽप्नोति तत्पदम्}

\twolineshloka
{भक्तियोगस्तदर्थी चेत् समग्रैश्वर्यसाधकः}
{आत्मार्थी चेत् त्रयोऽप्येते तत्कैवल्यस्य साधकाः}

\twolineshloka
{एकान्त्यं भगवत्येषां समानमधिकारिणाम्}
{यावत्प्राप्ति परार्थी चेत् तदेवात्यन्तमश्नुते}

\twolineshloka
{ज्ञानी तु परमैकन्ती तदायत्तत्मजीवनः}
{तत्संश्लेषवियोगैकसुखदुःखस्तदेकधीः}

\twolineshloka
{भगवद्\mbox{}ध्यानयोगोक्तिवन्दनस्तुतिकीर्तनैः}
{लब्धात्मा तन्दतप्राणमनोबुद्धीन्द्रियक्रियाः}

\twolineshloka
{निजकर्मादि भक्त्यन्तं कुर्यात् प्रीत्यैव कारितः}
{उपायतां परित्यज्य न्यस्येद्धेवे तु तामभीः}

\twolineshloka
{एकान्तात्यन्तदास्यैकरतिस्तत्पदमाप्नुयात्}
{तत्प्रधानमिदम् शास्त्रमिति गीतार्थसङ्ग्रहः}

॥इति श्री यामुनाचार्यविरचितं गीतर्थसङ्ग्रहः सम्पूर्णः॥
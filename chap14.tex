% !TeX program = XeLaTeX
% !TeX root = gitabook.tex
\sect{चतुर्दशोऽध्यायः}
{श्रीभगवानुवाच}
\twolineshloka
{परं भूयः प्रवक्ष्यामि ज्ञानानां ज्ञानमुत्तमम्}
{यज्ज्ञात्वा मुनयः सर्वे परां सिद्धिमितो गताः}% .. 14-1

\twolineshloka
{इदं ज्ञानमुपाश्रित्य मम साधर्म्यमागताः}
{सर्गेऽपि नोपजायन्ते प्रलये न व्यथन्ति च}% .. 14-2

\twolineshloka
{मम योनिर्महद्-ब्रह्म तस्मिन्गर्भं दधाम्यहम्}
{सम्भवः सर्वभूतानां ततो भवति भारत}% .. 14-3

\twolineshloka
{सर्वयोनिषु कौन्तेय मूर्तयः सम्भवन्ति याः}
{तासां ब्रह्म महद्योनिरहं बीजप्रदः पिता}% .. 14-4

\twolineshloka
{सत्त्वं रजस्तम इति गुणाः प्रकृतिसम्भवाः}
{निबध्नन्ति महाबाहो देहे देहिनमव्ययम्}% .. 14-5

\twolineshloka
{तत्र सत्त्वं निर्मलत्वात् प्रकाशकमनामयम्}
{सुखसङ्गेन बध्नाति ज्ञानसङ्गेन चानघ}% .. 14-6

\twolineshloka
{रजो रागात्मकं विद्धि तृष्णासङ्गसमुद्भवम्}
{तन्निबध्नाति कौन्तेय कर्मसङ्गेन देहिनम्}% .. 14-7

\twolineshloka
{तमस्त्वज्ञानजं विद्धि मोहनं सर्वदेहिनाम्}
{प्रमादालस्यनिद्राभिस्तन्निबध्नाति भारत}% .. 14-8

\twolineshloka
{सत्त्वं सुखे सञ्जयति रजः कर्मणि भारत}
{ज्ञानमावृत्य तु तमः प्रमादे सञ्जयत्युत}% .. 14-9

\twolineshloka
{रजस्तमश्चाभिभूय सत्त्वं भवति भारत}
{रजः सत्त्वं तमश्चैव तमः सत्त्वं रजस्तथा}% .. 14-10

\twolineshloka
{सर्वद्वारेषु देहेऽस्मिन् प्रकाश उपजायते}
{ज्ञानं यदा तदा विद्याद्विवृद्धं सत्त्वमित्युत}% .. 14-11

\twolineshloka
{लोभः प्रवृत्तिरारम्भः कर्मणामशमः स्पृहा}
{रजस्येतानि जायन्ते विवृद्धे भरतर्षभ}% .. 14-12

\twolineshloka
{अप्रकाशोऽप्रवृत्तिश्च प्रमादो मोह एव च}
{तमस्येतानि जायन्ते विवृद्धे कुरुनन्दन}% .. 14-13

\twolineshloka
{यदा सत्त्वे प्रवृद्धे तु प्रलयं याति देहभृत्}
{तदोत्तमविदां लोकानमलान् प्रतिपद्यते}% .. 14-14

\twolineshloka
{रजसि प्रलयं गत्वा कर्मसङ्गिषु जायते}
{तथा प्रलीनस्तमसि मूढयोनिषु जायते}% .. 14-15

\twolineshloka
{कर्मणः सुकृतस्यऽऽहुः सात्त्विकं निर्मलं फलम्}
{रजसस्तु फलं दुःखमज्ञानं तमसः फलम्}% .. 14-16

\twolineshloka
{सत्त्वात्सञ्जायते ज्ञानं रजसो लोभ एव च}
{प्रमादमोहौ तमसो भवतोऽज्ञानमेव च}% .. 14-17

\twolineshloka
{ऊर्ध्वं गच्छन्ति सत्त्वस्था मध्ये तिष्ठन्ति राजसाः}
{जघन्यगुणवृत्तिस्था अधो गच्छन्ति तामसाः}% .. 14-18

\twolineshloka
{नान्यं गुणेभ्यः कर्तारं यदा द्रष्टाऽनुपश्यति}
{गुणेभ्यश्च परं वेत्ति मद्भावं सोऽधिगच्छति}% .. 14-19

\twolineshloka
{गुणानेतानतीत्य त्रीन् देही देहसमुद्भवान्}
{जन्ममृत्युजरादुःखैर्विमुक्तोऽमृतमश्नुते}% .. 14-20

{अर्जुन उवाच}
\twolineshloka
{कैर्लिङ्गैस्त्रीन् गुणानेतानतीतो भवति प्रभो}
{किमाचारः कथं चैतांस्त्रीन् गुणानतिवर्तते}% .. 14-21

{श्रीभगवानुवाच}
\twolineshloka
{प्रकाशं च प्रवृत्तिं च मोहमेव च पाण्डव}
{न द्वेष्टि सम्प्रवृत्तानि न निवृत्तानि काङ्क्षति}% .. 14-22

\twolineshloka
{उदासीनवदासीनो गुणैर्यो न विचाल्यते}
{गुणा वर्तन्त इत्येव योऽवतिष्ठति नेङ्गते}% .. 14-23

\twolineshloka
{समदुःखसुखः स्वस्थः समलोष्टाश्मकाञ्चनः}
{तुल्यप्रियाप्रियो धीरस्तुल्यनिन्दात्मसंस्तुतिः}% .. 14-24

\twolineshloka
{मानापमानयोस्तुल्यस्तुल्यो मित्रारिपक्षयोः}
{सर्वारम्भपरित्यागी गुणातीतः स उच्यते}% .. 14-25

\twolineshloka
{मां च योऽव्यभिचारेण भक्तियोगेन सेवते}
{स गुणान् समतीत्यैतान् ब्रह्मभूयाय कल्पते}% .. 14-26

\twolineshloka
{ब्रह्मणो हि प्रतिष्ठाऽहममृतस्याव्ययस्य च}
{शाश्वतस्य च धर्मस्य सुखस्यैकान्तिकस्य च}% .. 14-27
{॥ॐ तत्सदिति श्रीमद्भगवद्गीतासूपनिषत्सु ब्रह्मविद्यायां योगशास्त्रे श्रीकृष्णार्जुनसंवादे गुणत्रयविभागयोगो नाम चतुर्दशोऽध्यायः॥}
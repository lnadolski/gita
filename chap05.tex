% !TeX program = XeLaTeX
% !TeX root = gitabook.tex
\sect{पञ्चमोऽध्यायः}
{अर्जुन उवाच}
\twolineshloka
{सन्न्यासं कर्मणां कृष्ण पुनर्योगं च शंससि}
{यच्छ्रेय एतयोरेकं तन्मे ब्रूहि सुनिश्चितम्}% .. 5-1

{श्रीभगवानुवाच}
\twolineshloka
{सन्न्यासः कर्मयोगश्च निःश्रेयसकरावुभौ}
{तयोस्तु कर्मसन्न्यासात्कर्मयोगो विशिष्यते}% .. 5-2

\twolineshloka
{ज्ञेयः स नित्यसन्न्यासी यो न द्वेष्टि न काङ्क्षति}
{निर्द्वन्द्वो हि महाबाहो सुखं बन्धात् प्रमुच्यते}% .. 5-3

\twolineshloka
{साङ्ख्ययोगौ पृथग्बालाः प्रवदन्ति न पण्डिताः}
{एकमप्यास्थितः सम्यगुभयोर्विन्दते फलम्}% .. 5-4

\twolineshloka
{यत्साङ्ख्यैः प्राप्यते स्थानं तद्योगैरपि गम्यते}
{एकं साङ्ख्यं च योगं च यः पश्यति स पश्यति}% .. 5-5

\twolineshloka
{सन्न्यासस्तु महाबाहो दुःखमाप्तुमयोगतः}
{योगयुक्तो मुनिर्ब्रह्म नचिरेणाधिगच्छति}% .. 5-6

\twolineshloka
{योगयुक्तो विशुद्धात्मा विजितात्मा जितेन्द्रियः}
{सर्वभूतात्मभूतात्मा कुर्वन्नपि न लिप्यते}% .. 5-7

\twolineshloka
{नैव किञ्चित्करोमीति युक्तो मन्येत तत्त्ववित्}
{पश्यञ्शृण्वन्स्पृशञ्जिघ्रन्नश्नङ्गच्छन्स्वपन्श्वसन्}%.. 5-8

\twolineshloka
{प्रलपन्विसृजन्गृह्णन्नुन्मिषन्निमिषन्नपि}
{इन्द्रियाणीन्द्रियार्थेषु वर्तन्त इति धारयन्}% .. 5-9

\twolineshloka
{ब्रह्मण्याधाय कर्माणि सङ्गं त्यक्त्वा करोति यः}
{लिप्यते न स पापेन पद्मपत्रमिवाम्भसा}% .. 5-10

\twolineshloka
{कायेन मनसा बुद्‌ध्या केवलैरिन्द्रियैरपि}
{योगिनः कर्म कुर्वन्ति सङ्गं त्यक्त्वाऽऽत्मशुद्धये}% .. 5-11

\twolineshloka
{युक्तःकर्मफलं त्यक्त्वा शान्तिमाप्नोति नैष्ठिकीम्}
{अयुक्तः कामकारेण फले सक्तो निबध्यते}% .. 5-12

\twolineshloka
{सर्वकर्माणि मनसा सन्न्यस्यऽऽस्ते सुखं वशी}
{नवद्वारे पुरे देही नैव कुर्वन्न कारयन्}% .. 5-13

\twolineshloka
{न कर्तृत्वं न कर्माणि लोकस्य सृजति प्रभुः}
{न कर्मफलसंयोगं स्वभावस्तु प्रवर्तते}% .. 5-14

\twolineshloka
{नऽऽदत्ते कस्यचित् पापं न चैव सुकृतं विभुः}
{अज्ञानेनऽऽवृतं ज्ञानं तेन मुह्यन्ति जन्तवः}% .. 5-15

\twolineshloka
{ज्ञानेन तु तदज्ञानं येषां नाशितमात्मनः}
{तेषामादित्यवज्ज्ञानं प्रकाशयति तत्परम्}% .. 5-16

\twolineshloka
{तद्बुद्धयस्तदात्मानस्तन्निष्ठास्तत्परायणाः}
{गच्छन्त्यपुनरावृत्तिं ज्ञाननिर्धूतकल्मषाः}% .. 5-17

\twolineshloka
{विद्याविनयसम्पन्ने ब्राह्मणे गवि हस्तिनि}
{शुनि चैव श्वपाके च पण्डिताः समदर्शिनः}% .. 5-18

\twolineshloka
{इहैव तैर्जितः सर्गो येषां साम्ये स्थितं मनः}
{निर्दोषं हि समं ब्रह्म तस्माद्-ब्रह्मणि ते स्थिताः}% .. 5-19

\twolineshloka
{न प्रहृष्येत् प्रियं प्राप्य नोद्विजेत् प्राप्य चाप्रियम्}
{स्थिरबुद्धिरसम्मूढो ब्रह्मविद्-ब्रह्मणि स्थितः}% .. 5-20

\twolineshloka
{बाह्यस्पर्शेष्वसक्तात्मा विन्दत्यात्मनि यत् सुखम्}
{स ब्रह्मयोगयुक्तात्मा सुखमक्षयमश्नुते}% .. 5-21

\twolineshloka
{ये हि संस्पर्शजा भोगा दुःखयोनय एव ते}
{आद्यन्तवन्तः कौन्तेय न तेषु रमते बुधः}% .. 5-22

\twolineshloka
{शक्नोतीहैव यः सोढुं प्राक् शरीरविमोक्षणात्}
{कामक्रोधोद्भवं वेगं स युक्तः स सुखी नरः}% .. 5-23

\twolineshloka
{योऽन्तःसुखोऽन्तरारामस्तथाऽन्तर्ज्योतिरेव यः}
{स योगी ब्रह्मनिर्वाणं ब्रह्मभूतोऽधिगच्छति}% .. 5-24

\twolineshloka
{लभन्ते ब्रह्मनिर्वाणमृषयः क्षीणकल्मषाः}
{छिन्नद्वैधा यतात्मानः सर्वभूतहिते रताः}% .. 5-25

\twolineshloka
{कामक्रोधवियुक्तानां यतीनां यतचेतसाम्}
{अभितो ब्रह्मनिर्वाणं वर्तते विदितात्मनाम्}% .. 5-26

\twolineshloka
{स्पर्शान् कृत्वा बहिर्बाह्यांश्चक्षुश्चैवान्तरे भ्रुवोः}
{प्राणापानौ समौ कृत्वा नासाभ्यन्तरचारिणौ}% .. 5-27

\twolineshloka
{यतेन्द्रियमनोबुद्धिर्मुनिर्मोक्षपरायणः}
{विगतेच्छाभयक्रोधो यः सदा मुक्त एव सः}% .. 5-28

\twolineshloka
{भोक्तारं यज्ञतपसां सर्वलोकमहेश्वरम्}
{सुहृदं सर्वभूतानां ज्ञात्वा मां शान्तिमृच्छति}% .. 5-29
{॥ॐ तत्सदिति श्रीमद्भगवद्गीतासूपनिषत्सु ब्रह्मविद्यायां योगशास्त्रे श्रीकृष्णार्जुनसंवादे कर्मसन्न्यासयोगो नाम पञ्चमोऽध्यायः॥}
% !TeX program = XeLaTeX
% !TeX root = gitabook.tex
\sect{द्वितीयोऽध्यायः}
{सञ्जय उवाच}
\twolineshloka
{तं तथा कृपयाऽऽविष्टम् अश्रुपूर्णाकुलेक्षणम्}
{विषीदन्तमिदं वाक्यमुवाच मधुसूदनः}% .. 2-1

\begin{minipage}{\linewidth}
{श्रीभगवानुवाच}
\twolineshloka
{कुतस्त्वा कश्मलमिदं विषमे समुपस्थितम्}
{अनार्यजुष्टमस्वर्ग्यमकीर्तिकरमर्जुन}% .. 2-2
\end{minipage}

\twolineshloka
{क्लैब्यं मा स्म गमः पार्थ नैतत्त्वय्युपपद्यते}
{क्षुद्रं हृदयदौर्बल्यं त्यक्त्वोत्तिष्ठ परन्तप}% .. 2-3

{अर्जुन उवाच}
\twolineshloka
{कथं भीष्ममहं सङ्ख्ये द्रोणं च मधुसूदन}
{इषुभिः प्रतियोत्स्यामि पूजार्हावरिसूदन}% .. 2-4

\fourlineindentedshloka
{गुरूनहत्वा हि महानुभावान्}
{श्रेयो भोक्तुं भैक्ष्यमपीह लोके}
{हत्वाऽर्थकामांस्तु गुरूनिहैव}
{भुञ्जीय भोगान् रुधिरप्रदिग्धान्}% .. 2-5

\fourlineindentedshloka
{न चैतद्विद्मः कतरन्नो गरीयो}
{यद्वा जयेम यदि वा नो जयेयुः}
{यानेव हत्वा न जिजीविषामः}
{तेऽवस्थिताः प्रमुखे धार्तराष्ट्राः}% .. 2-6

\fourlineindentedshloka
{कार्पण्यदोषोपहतस्वभावः}
{पृच्छामि त्वां धर्मसम्मूढचेताः}
{यच्छ्रेयः स्यान्निश्चितं ब्रूहि तन्मे}
{शिष्यस्तेऽहं शाधि मां त्वां प्रपन्नम्}% .. 2-7

\fourlineindentedshloka
{न हि प्रपश्यामि ममापनुद्याद्-}
{यच्छोकमुच्छोषणमिन्द्रियाणाम्}
{अवाप्य भूमावसपत्नमृद्धम्}
{राज्यं सुराणामपि चऽऽधिपत्यम्}% .. 2-8

{सञ्जय उवाच}
\twolineshloka
{एवमुक्त्वा हृषीकेशं गुडाकेशः परन्तप}
{न योत्स्य इति गोविन्दमुक्त्वा तूष्णीं बभूव ह}% .. 2-9

\twolineshloka
{तमुवाच हृषीकेशः प्रहसन्निव भारत}
{सेनयोरुभयोर्मध्ये विषीदन्तमिदं वचः}% .. 2-10

{श्रीभगवानुवाच}
\twolineshloka
{अशोच्यानन्वशोचस्त्वं प्रज्ञावादांश्च भाषसे}
{गतासूनगतासूंश्च नानुशोचन्ति पण्डिताः}% .. 2-11

\twolineshloka
{न त्वेवाहं जातु नऽऽसं न त्वं नेमे जनाधिपाः}
{न चैव न भविष्यामः सर्वे वयमतः परम्}% .. 2-12

\twolineshloka
{देहिनोऽस्मिन् यथा देहे कौमारं यौवनं जरा}
{तथा देहान्तरप्राप्तिर्धीरस्तत्र न मुह्यति}% .. 2-13

\twolineshloka
{मात्रास्पर्शास्तु कौन्तेय शीतोष्णसुखदुःखदाः}
{आगमापायिनोऽनित्यास्तांस्तितिक्षस्व भारत}% .. 2-14

\twolineshloka
{यं हि न व्यथयन्त्येते पुरुषं पुरुषर्षभ}
{समदुःखसुखं धीरं सोऽमृतत्वाय कल्पते}% .. 2-15

\twolineshloka
{नासतो विद्यते भावो नाभावो विद्यते सतः}
{उभयोरपि दृष्टोऽन्तस्त्वनयोस्तत्त्वदर्शिभिः}% .. 2-16

\twolineshloka
{अविनाशि तु तद्विद्धि येन सर्वमिदं ततम्}
{विनाशमव्ययस्यास्य न कश्चित् कर्तुमर्हति}% .. 2-17

\twolineshloka
{अन्तवन्त इमे देहा नित्यस्योक्ताः शरीरिणः}
{अनाशिनोऽप्रमेयस्य तस्माद्युध्यस्व भारत}% .. 2-18

\twolineshloka
{य एनं वेत्ति हन्तारं यश्चैनं मन्यते हतम्}
{उभौ तौ न विजानीतो नायं हन्ति न हन्यते}% .. 2-19

\fourlineindentedshloka
{न जायते म्रियते वा कदाचित्}
{नायं भूत्वा भविता वा न भूयः}
{अजो नित्यः शाश्वतोऽयं पुराणो}
{न हन्यते हन्यमाने शरीरे}% .. 2-20

\twolineshloka
{वेदाविनाशिनं नित्यं य एनमजमव्ययम्}
{कथं स पुरुषः पार्थ कं घातयति हन्ति कम्}% .. 2-21

\fourlineindentedshloka
{वासांसि जीर्णानि यथा विहाय}
{नवानि गृह्णाति नरोऽपराणि}
{तथा शरीराणि विहाय जीर्णानि}
{अन्यानि संयाति नवानि देही}% .. 2-22

\twolineshloka
{नैनं छिन्दन्ति शस्त्राणि नैनं दहति पावकः}
{न चैनं क्लेदयन्त्यापो न शोषयति मारुतः}% .. 2-23

\twolineshloka
{अच्छेद्योऽयमदाह्योऽयमक्लेद्योऽशोष्य एव च}
{नित्यः सर्वगतः स्थाणुरचलोऽयं सनातनः}% .. 2-24

\twolineshloka
{अव्यक्तोऽयमचिन्त्योऽयमविकार्योऽयमुच्यते}
{तस्मादेवं विदित्वैनं नानुशोचितुमर्हसि}% .. 2-25

\twolineshloka
{अथ चैनं नित्यजातं नित्यं वा मन्यसे मृतम्}
{तथाऽपि त्वं महाबाहो नैवं शोचितुमर्हसि}% .. 2-26

\twolineshloka
{जातस्य हि ध्रुवो मृत्युर्ध्रुवं जन्म मृतस्य च}
{तस्मादपरिहार्येऽर्थे न त्वं शोचितुमर्हसि}% .. 2-27

\twolineshloka
{अव्यक्तादीनि भूतानि व्यक्तमध्यानि भारत}
{अव्यक्तनिधनान्येव तत्र का परिदेवना}% .. 2-28

\fourlineindentedshloka
{आश्चर्यवत्पश्यति कश्चिदेनम्}
{आश्चर्यवद्वदति तथैव चान्यः}
{आश्चर्यवच्चैनमन्यः शृणोति}
{श्रुत्वाऽप्येनं वेद न चैव कश्चित्}% .. 2-29

\twolineshloka
{देही नित्यमवध्योऽयं देहे सर्वस्य भारत}
{तस्मात् सर्वाणि भूतानि न त्वं शोचितुमर्हसि}% .. 2-30.. 

\twolineshloka
{स्वधर्ममपि चावेक्ष्य न विकम्पितुमर्हसि}
{धर्म्याद्धि युद्धाच्छ्रेयोऽन्यत् क्षत्रियस्य न विद्यते}% .. 2-31

\twolineshloka
{यदृच्छया चोपपन्नं स्वर्गद्वारमपावृतम्}
{सुखिनः क्षत्रियाः पार्थ लभन्ते युद्धमीदृशम्}% .. 2-32

\twolineshloka
{अथ चेत्त्वमिमं धर्म्यं सङ्ग्रामं न करिष्यसि}
{ततः स्वधर्मं कीर्तिं च हित्वा पापमवाप्स्यसि}% .. 2-33

\twolineshloka
{अकीर्तिं चापि भूतानि कथयिष्यन्ति तेऽव्ययाम्}
{सम्भावितस्य चाकीर्तिर्मरणादतिरिच्यते}% .. 2-34

\twolineshloka
{भयाद्रणादुपरतं मंस्यन्ते त्वां महारथाः}
{येषां च त्वं बहुमतो भूत्वा यास्यसि लाघवम्}% .. 2-35

\twolineshloka
{अवाच्यवादांश्च बहून् वदिष्यन्ति तवाहिताः}
{निन्दन्तस्तव सामर्थ्यं ततो दुःखतरं नु किम्}% .. 2-36

\twolineshloka
{हतो वा प्राप्स्यसि स्वर्गं जित्वा वा भोक्ष्यसे महीम्}
{तस्मादुत्तिष्ठ कौन्तेय युद्धाय कृतनिश्चयः}% .. 2-37

\twolineshloka
{सुखदुःखे समे कृत्वा लाभालाभौ जयाजयौ}
{ततो युद्धाय युज्यस्व नैवं पापमवाप्स्यसि}% .. 2-38

\twolineshloka
{एषा तेऽभिहिता साङ्ख्ये बुद्धिर्योगे त्विमां शृणु}
{बुद्‍ध्या युक्तो यया पार्थ कर्मबन्धं प्रहास्यसि}% .. 2-39

\twolineshloka
{नेहाभिक्रमनाशोऽस्ति प्रत्यवायो न विद्यते}
{स्वल्पमप्यस्य धर्मस्य त्रायते महतो भयात्}% .. 2-40

\twolineshloka
{व्यवसायात्मिका बुद्धिरेकेह कुरुनन्दन}
{बहुशाखा ह्यनन्ताश्च बुद्धयोऽव्यवसायिनाम्}% .. 2-41

\twolineshloka
{यामिमां पुष्पितां वाचं प्रवदन्त्यविपश्चितः}
{वेदवादरताः पार्थ नान्यदस्तीति वादिनः}% .. 2-42

\twolineshloka
{कामात्मानः स्वर्गपरा जन्मकर्मफलप्रदाम्}
{क्रियाविशेषबहुलां भोगैश्वर्यगतिं प्रति}% .. 2-43

\twolineshloka
{भोगैश्वर्यप्रसक्तानां तयाऽपहृतचेतसाम्}
{व्यवसायात्मिका बुद्धिः समाधौ न विधीयते}% .. 2-44

\twolineshloka
{त्रैगुण्यविषया वेदा निस्त्रैगुण्यो भवार्जुन}
{निर्द्वन्द्वो नित्यसत्त्वस्थो निर्योगक्षेम आत्मवान्}% .. 2-45

\twolineshloka
{यावानर्थ उदपाने सर्वतः सम्प्लुतोदके}
{तावान् सर्वेषु वेदेषु ब्राह्मणस्य विजानतः}% .. 2-46

\twolineshloka
{कर्मण्येवाधिकारस्ते मा फलेषु कदाचन}
{मा कर्मफलहेतुर्भूर्मा ते सङ्गोऽस्त्वकर्मणि}% .. 2-47

\twolineshloka
{योगस्थः कुरु कर्माणि सङ्गं त्यक्त्वा धनञ्जय}
{सिद्‌ध्यसिद्‌ध्योः समो भूत्वा समत्वं योग उच्यते}% .. 2-48

\twolineshloka
{दूरेण ह्यवरं कर्म बुद्धियोगाद्धनञ्जय}
{बुद्धौ शरणमन्विच्छ कृपणाः फलहेतवः}% .. 2-49

\twolineshloka
{बुद्धियुक्तो जहातीह उभे सुकृतदुष्कृते}
{तस्माद्योगाय युज्यस्व योगः कर्मसु कौशलम्}% .. 2-50

\twolineshloka
{कर्मजं बुद्धियुक्ता हि फलं त्यक्त्वा मनीषिणः}
{जन्मबन्धविनिर्मुक्ताः पदं गच्छन्त्यनामयम्}% .. 2-51

\twolineshloka
{यदा ते मोहकलिलं बुद्धिर्व्यतितरिष्यति}
{तदा गन्तासि निर्वेदं श्रोतव्यस्य श्रुतस्य च}% .. 2-52

\twolineshloka
{श्रुतिविप्रतिपन्ना ते यदा स्थास्यति निश्चला}
{समाधावचला बुद्धिस्तदा योगमवाप्स्यसि}% .. 2-53

{अर्जुन उवाच}
\twolineshloka
{स्थितप्रज्ञस्य का भाषा समाधिस्थस्य केशव}
{स्थितधीः किं प्रभाषेत किमासीत व्रजेत किम्}% .. 2-54

{श्रीभगवानुवाच}
\twolineshloka
{प्रजहाति यदा कामान् सर्वान् पार्थ मनोगतान्}
{आत्मन्येवऽऽत्मना तुष्टः स्थितप्रज्ञस्तदोच्यते}% .. 2-55

\twolineshloka
{दुःखेष्वनुद्विग्नमनाः सुखेषु विगतस्पृहः}
{वीतरागभयक्रोधः स्थितधीर्मुनिरुच्यते}% .. 2-56

\twolineshloka
{यः सर्वत्रानभिस्नेहस्तत् तत् प्राप्य शुभाशुभम्}
{नाभिनन्दति न द्वेष्टि तस्य प्रज्ञा प्रतिष्ठिता}% .. 2-57

\twolineshloka
{यदा संहरते चायं कूर्मोऽङ्गानीव सर्वशः}
{इन्द्रियाणीन्द्रियार्थेऽभ्यस्तस्य प्रज्ञा प्रतिष्ठिता}% .. 2-58

\twolineshloka
{विषया विनिवर्तन्ते निराहारस्य देहिनः}
{रसवर्जं रसोऽप्यस्य परं दृष्ट्वा निवर्तते}% .. 2-59

\twolineshloka
{यततो ह्यपि कौन्तेय पुरुषस्य विपश्चितः}
{इन्द्रियाणि प्रमाथीनि हरन्ति प्रसभं मनः}% .. 2-60

\twolineshloka
{तानि सर्वाणि संयम्य युक्त आसीत मत्परः}
{वशे हि यस्येन्द्रियाणि तस्य प्रज्ञा प्रतिष्ठिता}% .. 2-61

\twolineshloka
{ध्यायतो विषयान् पुंसः सङ्गस्तेषूपजायते}
{सङ्गात् सञ्जायते कामः कामात् क्रोधोऽभिजायते}% .. 2-62

\twolineshloka
{क्रोधाद्भवति सम्मोहः सम्मोहात् स्मृतिविभ्रमः}
{स्मृतिभ्रंशाद्-बुद्धिनाशो बुद्धिनाशात् प्रणश्यति}% .. 2-63

\twolineshloka
{रागद्वेषवियुक्तैस्तु विषयानिन्द्रियैश्चरन्}
{आत्मवश्यैर्विधेयात्मा प्रसादमधिगच्छति}% .. 2-64

\twolineshloka
{प्रसादे सर्वदुःखानां हानिरस्योपजायते}
{प्रसन्नचेतसो ह्याशु बुद्धिः पर्यवतिष्ठते}% .. 2-65

\twolineshloka
{नास्ति बुद्धिरयुक्तस्य न चायुक्तस्य भावना}
{न चाभावयतः शान्तिरशान्तस्य कुतः सुखम्}% .. 2-66

\twolineshloka
{इन्द्रियाणां हि चरतां यन्मनोऽनुविधीयते}
{तदस्य हरति प्रज्ञां वायुर्नावमिवाम्भसि}% .. 2-67

\twolineshloka
{तस्माद्यस्य महाबाहो निगृहीतानि सर्वशः}
{इन्द्रियाणीन्द्रियार्थेभ्यस्तस्य प्रज्ञा प्रतिष्ठिता}% .. 2-68

\twolineshloka
{या निशा सर्वभूतानां तस्यां जागर्ति संयमी}
{यस्यां जाग्रति भूतानि सा निशा पश्यतो मुनेः}% .. 2-69

\fourlineindentedshloka
{आपूर्यमाणमचलप्रतिष्ठम्}
{समुद्रमापः प्रविशन्ति यद्वत्}
{तद्वत्कामा यं प्रविशन्ति सर्वे}
{स शान्तिमाप्नोति न कामकामी}% .. 2-70

\twolineshloka
{विहाय कामान् यः  सर्वान् पुमांश्चरति निःस्पृहः}
{निर्ममो निरहङ्कारः स शान्तिमधिगच्छति}% .. 2-71

\twolineshloka
{एषा ब्राह्मी स्थितिः पार्थ नैनां प्राप्य विमुह्यति}
{स्थित्वाऽस्यामन्तकालेऽपि ब्रह्मनिर्वाणमृच्छति}% .. 2-72
{॥ॐ तत्सदिति श्रीमद्भगवद्गीतासूपनिषत्सु ब्रह्मविद्यायां योगशास्त्रे श्रीकृष्णार्जुनसंवादे साङ्ख्ययोगो नाम द्वितीयोऽध्यायः॥}
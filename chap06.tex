% !TeX program = XeLaTeX
% !TeX root = gitabook.tex
\sect{षष्ठोऽध्यायः}
{श्रीभगवानुवाच}
\twolineshloka
{अनाश्रितः कर्मफलं कार्यं कर्म करोति यः}
{स सन्न्यासी च योगी च न निरग्निर्न चाक्रियः}% .. 6-1

\twolineshloka
{यं सन्न्यासमिति प्राहुर्योगं तं विद्धि पाण्डव}
{न ह्यसन्न्यस्तसङ्कल्पो योगी भवति कश्चन}% .. 6-2

\twolineshloka
{आरुरुक्षोर्मुनेर्योगं कर्म कारणमुच्यते}
{योगारूढस्य तस्यैव शमः कारणमुच्यते}% .. 6-3

\twolineshloka
{यदा हि नेन्द्रियार्थेषु न कर्मस्वनुषज्जते}
{सर्वसङ्कल्पसन्न्यासी योगारूढस्तदोच्यते}% .. 6-4

\twolineshloka
{उद्धरेदात्मनऽऽत्मानं नऽऽत्मानमवसादयेत्}
{आत्मैव ह्यात्मनो बन्धुरात्मैव रिपुरात्मनः}% .. 6-5

\twolineshloka
{बन्धुरात्माऽऽत्मनस्तस्य येनऽऽत्मैवऽऽत्मना जितः}
{अनात्मनस्तु शत्रुत्वे वर्तेतऽऽत्मैव शत्रुवत्}% .. 6-6

\twolineshloka
{जितात्मनः प्रशान्तस्य परमात्मा समाहितः}
{शीतोष्णसुखदुःखेषु तथा मानापमानयोः}% .. 6-7

\twolineshloka
{ज्ञानविज्ञानतृप्तात्मा कूटस्थो विजितेन्द्रियः}
{युक्त इत्युच्यते योगी समलोष्टाश्मकाञ्चनः}% .. 6-8

\twolineshloka
{सुहृन् मित्रार्युदासीनमध्यस्थद्वेष्यबन्धुषु}
{साधुष्वपि च पापेषु समबुद्धिर्विशिष्यते}% .. 6-9

\twolineshloka
{योगी युञ्जीत सततमात्मानं रहसि स्थितः}
{एकाकी यतचित्तात्मा निराशीरपरिग्रहः}% .. 6-10

\twolineshloka
{शुचौ देशे प्रतिष्ठाप्य स्थिरमासनमात्मनः}
{नात्युच्छ्रितं नातिनीचं चैलाजिनकुशोत्तरम्}% .. 6-11

\twolineshloka
{तत्रैकाग्रं मनः कृत्वा यतचित्तेन्द्रियक्रियाः}
{उपविश्यऽऽसने युञ्ज्याद्योगमात्मविशुद्धये}% .. 6-12

\twolineshloka
{समं कायशिरोग्रीवं धारयन्नचलं स्थिरः}
{सम्प्रेक्ष्य नासिकाग्रं स्वं दिशश्चानवलोकयन्}% .. 6-13

\twolineshloka
{प्रशान्तात्मा विगतभीर्ब्रह्मचारिव्रते स्थितः}
{मनः संयम्य मच्चित्तो युक्त आसीत मत्परः}% .. 6-14

\twolineshloka
{युञ्जन्नेवं सदाऽऽत्मानं योगी नियतमानसः}
{शान्तिं निर्वाणपरमां मत्संस्थामधिगच्छति}% .. 6-15

\twolineshloka
{नात्यश्नतस्तु योगोऽस्ति न चैकान्तमनश्नतः}
{न चातिस्वप्नशीलस्य जाग्रतो नैव चार्जुन}% .. 6-16

\twolineshloka
{युक्ताहारविहारस्य युक्तचेष्टस्य कर्मसु}
{युक्तस्वप्नावबोधस्य योगो भवति दुःखहा}% .. 6-17

\twolineshloka
{यदा विनियतं चित्तमात्मन्येवावतिष्ठते}
{निःस्पृहः सर्वकामेभ्यो युक्त इत्युच्यते तदा}% .. 6-18

\twolineshloka
{यथा दीपो निवातस्थो नेङ्गते सोपमा स्मृता}
{योगिनो यतचित्तस्य युञ्जतो योगमात्मनः}% .. 6-19

\twolineshloka
{यत्रोपरमते चित्तं निरुद्धं योगसेवया}
{यत्र चैवऽऽत्मनाऽऽत्मानं पश्यन्नात्मनि तुष्यति}% .. 6-20

\twolineshloka
{सुखमात्यन्तिकं यत्तद्-बुद्धिग्राह्यमतीन्द्रियम्}
{वेत्ति यत्र न चैवायं स्थितश्चलति तत्त्वतः}% .. 6-21

\twolineshloka
{यं लब्ध्वा चापरं लाभं मन्यते नाधिकं ततः}
{यस्मिन् स्थितो न दुःखेन गुरुणाऽपि विचाल्यते}% .. 6-22

\twolineshloka
{तं विद्याद्-दुःखसंयोगवियोगं योगसंज्ञितम्}
{स निश्चयेन योक्तव्यो योगोऽनिर्विण्णचेतसा}% .. 6-23

\twolineshloka
{सङ्कल्पप्रभवान् कामांस्त्यक्त्वा सर्वानशेषतः}
{मनसैवेन्द्रियग्रामं विनियम्य समन्ततः}% .. 6-24

\twolineshloka
{शनैः शनैरुपरमेद्-बुद्‌ध्या धृतिगृहीतया}
{आत्मसंस्थं मनः कृत्वा न किञ्चिदपि चिन्तयेत्}% .. 6-25

\twolineshloka
{यतो यतो निश्चरति मनश्चञ्चलमस्थिरम्}
{ततस्ततो नियम्यैतदात्मन्येव वशं नयेत्}% .. 6-26

\twolineshloka
{प्रशान्तमनसं ह्येनं योगिनं सुखमुत्तमम्}
{उपैति शान्तरजसं ब्रह्मभूतमकल्मषम्}% .. 6-27

\twolineshloka
{युञ्जन्नेवं सदाऽऽत्मानं योगी विगतकल्मषः}
{सुखेन ब्रह्मसंस्पर्शमत्यन्तं सुखमश्नुते}% .. 6-28

\twolineshloka
{सर्वभूतस्थमात्मानं सर्वभूतानि चऽऽत्मनि}
{ईक्षते योगयुक्तात्मा सर्वत्र समदर्शनः}% .. 6-29

\twolineshloka
{यो मां पश्यति सर्वत्र सर्वं च मयि पश्यति}
{तस्याहं न प्रणश्यामि स च मे न प्रणश्यति}% .. 6-30

\twolineshloka
{सर्वभूतस्थितं यो मां भजत्येकत्वमास्थितः}
{सर्वथा वर्तमानोऽपि स योगी मयि वर्तते}% .. 6-31

\twolineshloka
{आत्मौपम्येन सर्वत्र समं पश्यति योऽर्जुन}
{सुखं वा यदि वा दुःखं स योगी परमो मतः}% .. 6-32

{अर्जुन उवाच}
\twolineshloka
{योऽयं योगस्त्वया प्रोक्तः साम्येन मधुसूदन}
{एतस्याहं न पश्यामि चञ्चलत्वात् स्थितिं स्थिराम्}% .. 6-33

\twolineshloka
{चञ्चलं हि मनः कृष्ण प्रमाथि बलवद्-दृढम्}
{तस्याहं निग्रहं मन्ये वायोरिव सुदुष्करम्}% .. 6-34

{श्रीभगवानुवाच}
\twolineshloka
{असंशयं महाबाहो मनो दुर्निग्रहं चलम्}
{अभ्यासेन तु कौन्तेय वैराग्येण च गृह्यते}% .. 6-35

\twolineshloka
{असंयतात्मना योगो दुष्प्राप इति मे मतिः}
{वश्यात्मना तु यतता शक्योऽवाप्तुमुपायतः}% .. 6-36

{अर्जुन उवाच}
\twolineshloka
{अयतिः श्रद्धयोपेतो योगाच्चलितमानसः}
{अप्राप्य योगसंसिद्धिं कां गतिं कृष्ण गच्छति}% .. 6-37

\twolineshloka
{कच्चिन्नोभयविभ्रष्टश्छिन्नाभ्रमिव नश्यति}
{अप्रतिष्ठो महाबाहो विमूढो ब्रह्मणः पथि}% .. 6-38

\twolineshloka
{एतन्मे संशयं कृष्ण छेत्तुमर्हस्यशेषतः}
{त्वदन्यः संशयस्यास्य छेत्ता न ह्युपपद्यते}% .. 6-39

{श्रीभगवानुवाच}
\twolineshloka
{पार्थ नैवेह नामुत्र विनाशस्तस्य विद्यते}
{न हि कल्याणकृत् कश्चिद्-दुर्गतिं तात गच्छति}% .. 6-40

\twolineshloka
{प्राप्य पुण्यकृतां लोकानुषित्वा शाश्वतीः समाः}
{शुचीनां श्रीमतां गेहे योगभ्रष्टोऽभिजायते}% .. 6-41

\twolineshloka
{अथवा योगिनामेव कुले भवति धीमताम्}
{एतद्धि दुर्लभतरं लोके जन्म यदीदृशम्}% .. 6-42

\twolineshloka
{तत्र तं बुद्धिसंयोगं लभते पौर्वदेहिकम्}
{यतते च ततो भूयः संसिद्धौ कुरुनन्दन}% .. 6-43

\twolineshloka
{पूर्वाभ्यासेन तेनैव ह्रियते ह्यवशोऽपि सः}
{जिज्ञासुरपि योगस्य शब्दब्रह्मातिवर्तते}% .. 6-44

\twolineshloka
{प्रयत्नाद्यतमानस्तु योगी संशुद्धकिल्बिषः}
{अनेकजन्मसंसिद्धस्ततो याति परां गतिम्}% .. 6-45

\twolineshloka
{तपस्विभ्योऽधिको योगी ज्ञानिभ्योऽपि मतोऽधिकः}
{कर्मिभ्यश्चाधिको योगी तस्माद्योगी भवार्जुन}% .. 6-46

\twolineshloka
{योगिनामपि सर्वेषां मद्गतेनान्तरात्मना}
{श्रद्धावान् भजते यो मां स मे युक्ततमो मतः}% .. 6-47
{॥ॐ तत्सदिति श्रीमद्भगवद्गीतासूपनिषत्सु ब्रह्मविद्यायां योगशास्त्रे श्रीकृष्णार्जुनसंवादे आत्मसंयमयोगो नाम षष्ठोऽध्यायः॥}
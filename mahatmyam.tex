% !TeX program = XeLaTeX
% !TeX root = gitabook.tex
\sect{गीतामाहात्म्यम्}
{धरोवाच}
\twolineshloka
{भगवन् परेमेशान भक्तिरव्यभिचारिणी}
{प्रारब्धं भुज्यमानस्य कथं भवति हे प्रभो}% .. 1..

{श्री विष्णुरुवाच}
\twolineshloka
{प्रारब्धं भुज्यमानो हि गीताभ्यासरतः सदा}
{स मुक्तः स सुखी लोके कर्मणा नोपलिप्यते}% .. 2..

\twolineshloka
{महापापादिपापानि गीताध्यानं करोति चेत्}
{क्वचित् स्पर्शं न कुर्वन्ति नलिनीदलमम्बुवत्}% .. 3..

\twolineshloka
{गीतायाः पुस्तकं यत्र यत्र पाठः प्रवर्तते}
{तत्र सर्वाणि तीर्थानि प्रयागादीनि तत्र वै}% .. 4..

\threelineshloka
{सर्वे देवाश्च ऋषयो योगिनः पन्नगाश्च ये}
{गोपाला गोपिका वाऽपि नारदोद्धवपार्षदैः}
{सहायो जायते शीघ्रं यत्र गीता प्रवर्तते}% .. 5..

\twolineshloka
{यत्र गीताविचारश्च पठनं पाठनं श्रुतम्}
{तत्राहं निश्चितं पृथ्वि निवसामि सदैव हि}% .. 6..

\twolineshloka
{गीताश्रयेऽहं तिष्ठामि गीता मे चोत्तमं गृहम्}
{गीताज्ञानमुपाश्रित्य त्रीँल्लोकान् पालयाम्यहम्}% .. 7..

\twolineshloka
{गीता मे परमा विद्या ब्रह्मरूपा न संशयः}
{अर्धमात्राक्षरा नित्या स्वानिर्वाच्यपदात्मिका}% .. 8..

\twolineshloka
{चिदानन्देन कृष्णेन प्रोक्ता स्वमुखतोऽर्जुनम्}
{वेदत्रयी परानन्दा तत्त्वार्थज्ञानसंयुता}% .. 9..

\twolineshloka
{योऽष्टादशजपो नित्यं नरो निश्चलमानसः}
{ज्ञानसिद्धिं स लभते ततो याति परं पदम्}% .. 10..

\twolineshloka
{पाठेऽसमर्थः सम्पूर्णे ततोऽर्धं पाठमाचरेत्}
{तदा गोदानजं पुण्यं लभते नात्र संशयः}% .. 11..

\twolineshloka
{त्रिभागं पठमानस्तु गङ्गास्नानफलं लभेत्}
{षडंशं जपमानस्तु सोमयागफलं लभेत्}% .. 12..

\twolineshloka
{एकाध्यायं तु यो नित्यं पठते भक्तिसंयुतः}
{रुद्रलोकमवाप्नोति गणो भूत्वा वसेच्चिरम्}% .. 13..

\twolineshloka
{अध्यायं श्लोकपादं वा नित्यं यः पठते नरः}
{स याति नरतां यावन्मन्वन्तरं वसुन्धरे}% .. 14..

\twolineshloka
{गीतायाः श्लोकदशकं सप्त पञ्च चतुष्टयम्}
{द्वौ त्रीनेकं तदर्धं वा श्लोकानां यः पठेन्नरः}% .. 15..

\twolineshloka
{चन्द्रलोकमवाप्नोति वर्षाणामयुतं ध्रुवम्}
{गीतापाठसमायुक्तो मृतो मानुषतां व्रजेत्}% .. 16..

\twolineshloka
{गीताभ्यासं पुनः कृत्वा लभते मुक्तिमुत्तमाम्}
{गीतेत्युच्चारसंयुक्तो म्रियमाणो गतिं लभेत्}% .. 17..

\twolineshloka
{गीतार्थश्रवणासक्तो महापापयुतोऽपि वा}
{वैकुण्ठं समवाप्नोति विष्णुना सह मोदते}% .. 18..

\twolineshloka
{गीतार्थं ध्यायते नित्यं कृत्वा कर्माणि भूरिशः}
{जीवन्मुक्तः स विज्ञेयो देहान्ते परमं पदम्}% .. 19..

\twolineshloka
{गीतामाश्रित्य बहवो भूभुजो जनकादयः}
{निर्धूतकल्मषा लोके गीतायाताः परं पदम्}% .. 20..

\twolineshloka
{गीतायाः पठनं कृत्वा माहात्म्यं नैव यः पठेत्}
{वृथा पाठो भवेत्तस्य श्रम एव ह्युदाहृतः}% .. 21..

\twolineshloka
{एतन्माहात्म्यसंयुक्तं गीताभ्यासं करोति यः}
{स तत् फलमवाप्नोति दुर्लभां गतिमाप्नुयात्}% .. 22..

{सूत उवाच}
\twolineshloka
{माहात्म्यमेतद्गीताया मया प्रोक्तं सनातनम्}
{गीतान्ते च पठेद्यस्तु यदुक्तं तत्फलं लभेत्}% .. 23..
{॥ इति श्रीवाराहपुराणे श्रीगीतामाहात्म्यं सम्पूर्णम्॥}
\clearpage
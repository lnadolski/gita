% !TeX program = XeLaTeX
% !TeX root = gitabook.tex
\sect{सप्तमोऽध्यायः}
\uvacha{श्रीभगवानुवाच}
\twolineshloka
{मय्यासक्तमनाः पार्थ योगं युञ्जन्मदाश्रयः}
{असंशयं समग्रं मां यथा ज्ञास्यसि तच्छृणु}% .. 7-1

\twolineshloka
{ज्ञानं तेऽहं सविज्ञानमिदं वक्ष्याम्यशेषतः}
{यज्ज्ञात्वा नेह भूयोऽन्यज्ज्ञातव्यमवशिष्यते}% .. 7-2

\twolineshloka
{मनुष्याणां सहस्रेषु कश्चिद्यतति सिद्धये}
{यततामपि सिद्धानां कश्चिन्मां वेत्ति तत्त्वतः}% .. 7-3

\twolineshloka
{भूमिरापोऽनलो वायुः खं मनो बुद्धिरेव च}
{अहङ्कार इतीयं मे भिन्ना प्रकृतिरष्टधा}% .. 7-4

\twolineshloka
{अपरेयमितस्त्वन्यां प्रकृतिं विद्धि मे पराम्}
{जीवभूतां महाबाहो ययेदं धार्यते जगत्}% .. 7-5

\twolineshloka
{एतद्योनीनि भूतानि सर्वाणीत्युपधारय}
{अहं कृत्स्नस्य जगतः प्रभवः प्रलयस्तथा}% .. 7-6

\twolineshloka
{मत्तः परतरं नान्यत् किञ्चिदस्ति धनञ्जय}
{मयि सर्वमिदं प्रोतं सूत्रे मणिगणा इव}% .. 7-7

\twolineshloka
{रसोऽहमप्सु कौन्तेय प्रभाऽस्मि शशिसूर्ययोः}
{प्रणवः सर्ववेदेषु शब्दः खे पौरुषं नृषु}% .. 7-8

\twolineshloka
{पुण्यो गन्धः पृथिव्यां च तेजश्चास्मि विभावसौ}
{जीवनं सर्वभूतेषु तपश्चास्मि तपस्विषु}% .. 7-9

\twolineshloka
{बीजं मां सर्वभूतानां विद्धि पार्थ सनातनम्}
{बुद्धिर्बुद्धिमतामस्मि तेजस्तेजस्विनामहम्}% .. 7-10

\twolineshloka
{बलं बलवतां चाहं कामरागविवर्जितम्}
{धर्माविरुद्धो भूतेषु कामोऽस्मि भरतर्षभ}% .. 7-11

\twolineshloka
{ये चैव सात्त्विका भावा राजसास्तामसाश्च ये}
{मत्त एवेति तान् विद्धि न त्वहं तेषु ते मयि}% .. 7-12

\twolineshloka
{त्रिभिर्गुणमयैर्भावैरेभिः सर्वमिदं जगत्}
{मोहितं नाभिजानाति मामेभ्यः परमव्ययम्}% .. 7-13

\twolineshloka
{दैवी ह्येषा गुणमयी मम माया दुरत्यया}
{मामेव ये प्रपद्यन्ते मायामेतां तरन्ति ते}% .. 7-14

\twolineshloka
{न मां दुष्कृतिनो मूढाः प्रपद्यन्ते नराधमाः}
{माययाऽपहृतज्ञाना आसुरं भावमाश्रिताः}% .. 7-15

\twolineshloka
{चतुर्विधा भजन्ते मां जनाः सुकृतिनोऽर्जुन}
{आर्तो जिज्ञासुरर्थार्थी ज्ञानी च भरतर्षभ}% .. 7-16

\twolineshloka
{तेषां ज्ञानी नित्ययुक्त एकभक्तिर्विशिष्यते}
{प्रियो हि ज्ञानिनोऽत्यर्थमहं स च मम प्रियः}% .. 7-17

\twolineshloka
{उदाराः सर्व एवैते ज्ञानी त्वात्मैव मे मतम्}
{आस्थितः स हि युक्तात्मा मामेवानुत्तमां गतिम्}% .. 7-18

\twolineshloka
{बहूनां जन्मनामन्ते ज्ञानवान् मां प्रपद्यते}
{वासुदेवः सर्वमिति स महात्मा सुदुर्लभः}% .. 7-19.. 

\twolineshloka
{कामैस्तैस्तैर्हृतज्ञानाः प्रपद्यन्तेऽन्यदेवताः}
{तं तं नियममास्थाय प्रकृत्या नियताः स्वया}% .. 7-20

\twolineshloka
{यो यो यां यां तनुं भक्तः श्रद्धयाऽर्चितुमिच्छति}
{तस्य तस्याचलां श्रद्धां तामेव विदधाम्यहम्}% .. 7-21.. 

\twolineshloka
{स तया श्रद्धया युक्तस्तस्यऽऽराधनमीहते}
{लभते च ततः कामान् मयैव विहितान् हि तान्}% .. 7-22

\twolineshloka
{अन्तवत्तु फलं तेषां तद्भवत्यल्पमेधसाम्}
{देवान् देवयजो यान्ति मद्भक्ता यान्ति मामपि}% .. 7-23

\twolineshloka
{अव्यक्तं व्यक्तिमापन्नं मन्यन्ते मामबुद्धयः}
{परं भावमजानन्तो ममाव्ययमनुत्तमम्}% .. 7-24

\twolineshloka
{नाहं प्रकाशः सर्वस्य योगमायासमावृतः}
{मूढोऽयं नाभिजानाति लोको मामजमव्ययम्}% .. 7-25

\twolineshloka
{वेदाहं समतीतानि वर्तमानानि चार्जुन}
{भविष्याणि च भूतानि मां तु वेद न कश्चन}% .. 7-26

\twolineshloka
{इच्छाद्वेषसमुत्थेन द्वन्द्वमोहेन भारत}
{सर्वभूतानि सम्मोहं सर्गे यान्ति परन्तप}% .. 7-27

\twolineshloka
{येषां त्वन्तगतं पापं जनानां पुण्यकर्मणाम्}
{ते द्वन्द्वमोहनिर्मुक्ता भजन्ते मां दृढव्रताः}% .. 7-28

\twolineshloka
{जरामरणमोक्षाय मामाश्रित्य यतन्ति ये}
{ते ब्रह्म तद्विदुः कृत्स्नमध्यात्मं कर्म चाखिलम्}% .. 7-29

\twolineshloka
{साधिभूताधिदैवं मां साधियज्ञं च ये विदुः}
{प्रयाणकालेऽपि च मां ते विदुर्युक्तचेतसः}% .. 7-30
{॥ॐ तत्सदिति श्रीमद्भगवद्गीतासूपनिषत्सु ब्रह्मविद्यायां योगशास्त्रे श्रीकृष्णार्जुनसंवादे ज्ञानविज्ञानयोगो नाम सप्तमोऽध्यायः॥}